%!TEX TS-program = xelatex
\documentclass[]{friggeri-cv}
% \documentclass[print]{friggeri-cv}

\begin{document}
\header{charles}{moraes}
       {computer science}


% In the aside, each new line forces a line break
\begin{aside}
  \section{about}
    \href{mailto:raverblazer@gmail.com}{raverblazer@gmail.com}
    \href{http://www.linkedin.com/in/charlesdm}{\faLinkedinSign \space  /in/charlesdm}
    \href{https://github.com/streeck}{\faGithubSign \space github.com/streeck}
    ~
    (11) 5066-7442
    (11) 99421-0220
    ~
    Avenida Moaci 1885
    Planalto Paulista
    04083-005
    São Paulo, SP
    Brazil
  \section{languages}
    native portuguese
    fluent english
    spanish notions
  \section{programming}
    Python, R
    Django, Numpy, Scipy
    Ruby on Rails
    C, C++, Java
    PostgreSQL
    JavaScript
    CSS3 \& HTML5
\end{aside}

\section{interests}

Data science, big data, information/network security, artificial intelligence, machine learning, data mining, innovation, new technologies

\section{education}

\begin{entrylist}
  \entry
    {2012 - 2017}
    {B.Sc.}
    {Federal University of São Carlos -- UFSCar, Sorocaba}
    {Majoring in Computer Science}
  \entry
    {2013 – 2014}
    {Exchange Program}
    {Bangor University -- Bangor, Gwynedd, United Kingdom}
    {Science Without Borders Program - fully funded scholarship recipient \\
    Non-graduating Undergraduate in Computer Sciences \\
    Non-degree seeking credits at Undergraduate Bachelor degree level}
\end{entrylist}

\section{volunteer}

\begin{entrylist}
  \entry
    {since 2015}
    {TECHO | TETO}
    {Volunteer}
    {Volunteer in building emergency houses for people in extreme poverty conditions.}
\end{entrylist}

\section{courses}

\begin{entrylist}
  \entry
    {2015}
    {Coursera}
    {Several Universities}
    {Data Analysis and Statistical Inference \\
    Big Data em Saúde no Brasil \\
    The Data Scientist's Toolbox \\
    R Programming}
  \entry
    {2015}
    {Reuso de Software através de Linhas de Produto de Software}
    {UFSCar}
    {Gerar produtos específicos com base no reúso de uma infraestrutura que contempla arquitetura de software, componentes, padrões de projeto e métodos de planejamento. \\
    Duração: 40 horas}
\end{entrylist}


\end{document}
